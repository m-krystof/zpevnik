\begin{TEXT}{Ach ach ach (hymna 2018)}

\SLOKA Když \Ch{A}{slyším} něčí bzučení a \Ch{D7}{žaludku pak kručení}\NL
tak \Ch{A}{vzpomenu} si \Ch{E}{ihned} na vče\Ch{A}{ly}\NL
Že \Ch{A}{včely} mají sladký med \Ch{D7}{a to proto} abych já ho sněd´\NL
to \Ch{A}{předpokládám} \Ch{E}{už} jste vědě\Ch{A}{li}\NL

Dostat se k němu těžká věc a čím víc myslím tím víc se\NL
tu vytrácí pak moje veselí\NL
Po čase pak už spadla klec já v pasti doufám nakonec\NL
že někdo ten můj balón sestřelí

\REFREN

\Ch{D7}{Ach ach ach tys tomu} \Ch{A7}{dal}\NL
\Ch{D7}{ach ach ach} ne\Ch{H7}{chtěj} abych se \Ch{E}{smál}\NL
/: Ale \Ch{A}{stačí} jenom \Ch{A7}{chtít} dobrou \Ch{D}{vůli} k tomu \Ch{D7}{mít}\NL
můžeš \Ch{A}{nad} blbostí \Ch{E}{přece} zvítě\Ch{A}{zit} :/

\SLOKA Že někdo nový přišel k nám to těžko nějak pobírám\NL
a jsem věřte z toho celý hotový\NL
Je divný cizí jinačí a ke štěstí nám postačí\NL
kdyby odtáhl zas pryč no to se ví

\REFREN Paf paf paf no tys ulít´\NL
         paf paf paf copak chceš nás otrávit?\NL
/: Není potřeba se bát\NL
 nechtěj dveře zavírat\NL
můžeš ve druhém bližního rozeznat :/

\SLOKA Kolik přátel tady vůbec mám já když se kolem podívám\NL
tak šedý jako já svět se mi zdá\NL
Málo má barev štěstí radosti i když jsem dobrák od kosti\NL
a smysl zdá se vůbec postrádá

\REFREN Aj aj aj no to snad ne\NL
aj aj aj to je přímo nešťastné\NL
/: Nemusíš se pořád smát\NL
jen se jinak podívat\NL
a pak mrzutosti dá se sbohem dát :/

\SLOKA Nejsme dokonalí to fakt ne a může se zdát podivné\NL
že vůbec takhle spolu žít se dá\NL
Kromě dobroty a radosti máme své mindráky a slabosti\NL
a náš svět podle toho vypadá

\REFREN Ach ach ach no ten náš svět\NL
ach ach ach no ten bych změnil hned\NL
Ale stačí jenom chtít\NL
Boží sílu k tomu mít\NL
můžeš sebe i svět k dobru proměnit\NL
/: Jenže nestačí jen snít\NL
chce to zvednout se a jít\NL
můžeš i někoho v nouzi zachránit :/\NL
\hspace*{2cm}--- JAKO PÚ !
\end{TEXT}