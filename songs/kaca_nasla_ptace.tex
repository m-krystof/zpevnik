\begin{TEXT}{Káča našla ptáče}
\SLOKA /: \Ch{D}{Káča} našla ptáče je \Ch{G}{promoklé} a \Ch{D}{pláče}
kdo poradí Káče \Ch{Em}{komu} ptáče \Ch{A}{dát} :/
/: Niko\Ch{D}{mu} jinné\Ch{A}{mu} jenom panu hajné\Ch{D}{mu}
on ho dá do hníz\Ch{A}{da} ptáče zase zahvíz\Ch{D}{dá} :/
\SLOKA /: Jířa našla zvíře co netrefilo k díře
kdo poradí Jíře komu zvíře dát :/
/: Nikomu jinému jenom panu hajnému 
hajnému jedině o ho vrátí rodině :/
\SLOKA /: Míra našel výra má po noční a zírá
zajímá se Míra komu výra dát :/
/: Nikomu jinému jenom panu hajnému 
Může mu radu dát kdy má lítat a kdy spát :/
\SLOKA /: Na dolejší lávce tam Slávka našla savce
kdo poradí Slávce komu savce dát :/
/: Vždyť je to Pavlata zadělaný od bláta
měli ho za zvíře špinavý byl k nevíře :/
\INTERM Teda Pavlata ty vypadáš
\end{TEXT}