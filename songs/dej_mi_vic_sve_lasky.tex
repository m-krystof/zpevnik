\begin{TEXT}{Dej mi víc své lásky}
\SLOKA \Ch{e}{Vymyslel} jsem spoustu napadů, \Ch{G}{aú,} \NL
co \Ch{e}{podporujou} hloupou nála\Ch{D}{du,} aú,\Ch{H7}{~}  \NL
\Ch{e}{hodit} klíče do kanálu, \Ch{A}{sjet} po zadku \Ch{a}{holou skálu,} \NL
\Ch{e}{v noci} chodit \Ch{H7}{strašit} do \Ch{e}{hradu.}
\SLOKA Dám si dvoje housle pod bradu, aú,\NL
v bílé plachtě chodím pozadu, aú,\NL
úplně melancholicky, s citem pro věc jako vždycky\NL
vyrábím tu hradní záhadu, aú.
\REFREN \Ch{G}{Má drahá,} dej mi víc, \Ch{H7}{má drahá, dej mi víc,}\NL
\Ch{e}{má drahá,} \Ch{C}{dej} mi víc své \Ch{G}{lásky,} \Ch{D7}{aú,}\NL
\Ch{G}{já nechci} skoro nic, \Ch{H7}{já nechci skoro nic,}\NL
\Ch{e}{já chci} jen \Ch{C}{pohladit} tvé \Ch{G}{vlásky,} \Ch{D7}{aú.}
\SLOKA Nejlepší z těch divnejch nápadů, aú,\NL
  mi dokonale zvednul náladu, aú,\NL
  natrhám ti sedmikrásky, tebe celou s tvými vlásky\NL
  zamknu si na sedm západů, aú, aú, aú. aú.
\REFRENHRAJ
\end{TEXT}
