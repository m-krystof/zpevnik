\begin{TEXT}{Balada velkopáteční}
\SLOKA \Ch{a}{Shromáždil} \Ch{E}{člověk} prý \Ch{C}{moudrosti} své \Ch{D}{poklad}\NL
\Ch{F}{uložil} jej \Ch{G}{do svých} příslo\Ch{a}{ví}\Ch{E}{}\NL
\Ch{a}{A kdo} chce \Ch{E}{pravdu} snad \Ch{C}{o životě} \Ch{D}{poznat}\NL
\Ch{F}{k tomu} ať ten \Ch{G}{poklad} promlu\Ch{a}{ví}\Ch{E}{}\NL
\Ch{A}{Chcete-li} však \Ch{A7}{takto} dojít \Ch{d}{poučení}\NL
\Ch{G}{nezůstaňte} v půli cesty \Ch{C}{stát}\Ch{E}{}\NL
\Ch{a}{Snažte} se \Ch{E}{naslouchat} \Ch{C}{jen} plnému \Ch{D}{znění}\NL
\Ch{F}{půlka} pravdy \Ch{G}{může} taky \Ch{a}{lhát}

\SLOKA Do jámy upadne kdo jiným ji chystá\NL
sejde mečem kdo s ním zachází\NL
Pravda však bývá pouze tehdy jistá\NL
když jí jedna půlka neschází\NL
Slyšte proto všichni druhou polovinu\NL
je v ní skryta moudrost odvěká:\NL
/: Ten kdo meč odmítá ten na kříži sejde\NL
slyšte lidé moudrost člověka :/
\end{TEXT}