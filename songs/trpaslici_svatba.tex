\begin{TEXT}{Trpasličí svatba}

\SLOKA \Ch{G}{V lese}, jó v lese \Ch{C}{na} jehličí \Ch{G}\NL
Koná se svatba \Ch{A}{trpasličí}\Ch{D7}\NL
\Ch{G}{Žádná} divná \Ch{C}{věc} to není \Ch{H7}\NL
\Ch{C}{Šmudla} \Ch{C#dim}{už má} \Ch{G}{po} voj\Ch{e}{ně} a tak se \Ch{D7}{žení}\Ch{G}
   
\SLOKA Malou má ženu, malinkatou,\NL
S malým věnem, malou chatou.\NL
Už jim choděj' telegrámky,\NL
už jim hrajou Mendelssohna na varhánky.\NL

\REFREN \Ch{G}{Protože} \Ch{D}{láska}, láska, láska v každém \Ch{G}{srdci} klí\Ch{D}{čí}\NL
\Ch{D}{Protože} láskou hoří i to srdce \Ch{G}{trpasli}\Ch{D}{čí}\NL
\Ch{D}{A kdo} se v téhle věci jednoduše \Ch{G}{neopi}\Ch{D}{čí}\NL
\Ch{D}{Ten ať} se dívá, co se děje v lese \Ch{G}{na} jehli\Ch{D}{čí} \Ch{D7}

\SLOKA Mají tam lásku, jako trámek\NL
Pláče tchýňka, pláče tchánek\NL
Štěstíčko přejou mladí, staří\NL
V papinově hrníčku se maso vaří

\SLOKA Pijou tam pivo popovický\NL
Šmudla se polil, jako vždycky\NL
Kejchal kejchá na Profouse\NL
Jedí hrášek s uzeným a nafouknou se

\REFRENHRAJ

\SLOKA V lese, jó v lese na jehličí\NL
Koná se svatba trpasličí\NL
Žádná divná věc to není\NL
Šmudla už má po vojně a tak se žení
\end{TEXT}