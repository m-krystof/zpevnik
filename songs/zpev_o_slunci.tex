\begin{TEXT}{Zpěv o slunci}
\SLOKA[Intr.] \Ch{G6}{Nejvyšší}, \Ch{D}{všemohoucí}, \Ch{A}{dobrý} Pa\Ch{E}{ne}\NL
ať \Ch{D}{Tobě} \Ch{h}{chvála}, sláva, \Ch{c#}{pocta} se sta\Ch{E}{ne},\NL
\Ch{A}{všeliké} \Ch{D}{dobro}\Ch{A}{ře}\Ch{E}{če}\Ch{A}{ní}.\NL
\SLOKA \Ch{A}{Buď} pochválen\Ch{D}{ }Pane můj\NL
se \Ch{A}{svým} stvořením \Ch{E}{po} vší zemi\NL        
\Ch{A}{se} sluncem jež\Ch{D}{ }činí den\NL
\Ch{A}{a} září nad \Ch{E}{lidmi} všemi\NL
\REFREN \Ch{A}{Chvalte} Pána, \Ch{D}{díky} vzdejte, \Ch{A}{služte} \Ch{E}{v po}ko\Ch{A}{ře}
\SLOKA Buď pochválen Pane můj\NL
od noční luny i hvězd krásných\NL
stvořil jsi jich na nebi\NL
tolik světlých a vždy jasných\NL
\SLOKA Buď pochválen Pane můj\NL
i vodou čistou s její krásou\NL
ohněm, který plál nocí,\NL
jitrem s jeho ranní rosou\NL
\SLOKA Buď pochválen Pane můj\NL
i rodnou zemí matku naší\NL
dává život, svět celý\NL
květem pestrých barev krášlí\NL
\SLOKA Buď pochválen Pane můj\NL
i sestrou naší – těla smrtí,\NL
nemá moc když od Tebe\NL
přijmou život tvoji svatí\NL
\SLOKA* (Nakonec znovu intr., 1. sloka a ref.) 
\end{TEXT}