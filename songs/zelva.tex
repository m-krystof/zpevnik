\begin{TEXT}{Želva}
\SLOKA \Ch{D}{Ne moc} \Ch{G}{snadno} se \Ch{D}{želva} \Ch{G}{po dně} ho\Ch{D G D G}{ní,}\NL
   \Ch{D}{velmi} \Ch{G}{radno} je \Ch{D}{plavat} \Ch{G}{na dno za} \Ch{D G D}{ní,}\NL
   potom \Ch{A}{počkej,} až se zeptá na to, \Ch{h}{co tě v mozku lechtá,}\NL
   \Ch{D}{nic se} \Ch{G}{neboj} a \Ch{D}{vem si} \Ch{G}{něco od} \Ch{D G D G}{ní.}

\SLOKA Abych zabil dvě mouchy jednou ranou,\NL
želví nervy od želvy schovám stranou,\NL
jednu káď tam dám pro sebe a pak aspoň pět pro tebe\NL
víš, má drahá, a zbytek je pod vanou.


\REFREN \Ch{D}{Když si} \Ch{A}{někdo} \Ch{G}{pozor} \Ch{D}{nedá,} jak se \Ch{A}{vlastně} \Ch{G}{želva} \Ch{D}{hledá,}\NL
   \Ch{G}{ona ho na} něco nachy\Ch{A}{tá,} \Ch{G}{i když si to} později vyčí\Ch{A  A7}{tá.}

\SLOKA Ne moc lehce se želva po dně honí,\NL
ten, kdo nechce, tak brzy slzy roní,\NL
jeho úsměv se vytratí, a to se mu nevyplatí,\NL
má se nebát želev a spousty vodní.

\REFRENHRAJ
\end{TEXT}
