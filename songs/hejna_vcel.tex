\begin{TEXT}{Hejna včel}
\SLOKA \Ch{a}{Nějak} umírá nám \Ch{a/G}{láska} \NL
my jako \Ch{a/F}{hejna} divejch \Ch{E}{včel} jdeme \Ch{a}{dál} \Ch{a/G}{ } \Ch{a/F}{ } \Ch{E}{ }
\SLOKA každej vztah je vlastně sázka \NL
každý ráno může zmizet                        my jdeme dál 
\REFREN  \Ch{a}{Řekně}te kdopak za to \Ch{a/G}{může} \NL
kdo z nás má \Ch{a/F}{právo} něco br\Ch{E}{át } \NL
\Ch{a}{kdo} učil lidi zlobu \Ch{a/G}{dýchat} \NL
kdo na vo\Ch{a/F}{jáky} chce si \Ch{E}{hrát}
\SLOKA Už zase bohatejch je spousta \NL
a čím víc peněz lásky míň                    jdeme dál
\SLOKA A všichni plnou pusu slov \NL
že prý už zítra zítra snad                    budeme dál
\REFREN 
\SLOKA Už zase umírá nám láska \NL
my jako hejna divejch včel \NL
jdeme dál…
\end{TEXT}
