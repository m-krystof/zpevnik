\begin{TEXT}{Kdyby se v komnatách}
\SLOKA \Ch{C}{K životu} na \Ch{Gmi}{zámku}, mám jednu poznámku,\NL
   \Ch{F}{Je tu neveselo,} \Ch{C}{je tu truchlivo.}\NL
   V ostatních královs\Ch{Gmi}{tvích,} nezní tak málo smích,\NL
   \Ch{F}{Není} neveselo, \Ch{C}{není} truchlivo.
\REFREN \Ch{C7}{Kdyby} \Ch{F}{se v komnatách,} \Ch{Emi}{běhoun jak hrom natáh,}\NL
   a na \Ch{Dmi}{něm} akro\Ch{G}{bati,} za\Ch{C}{čali kejklovati.}\NL
   \Ch{C7}{Kdyby} \Ch{F}{nám v paláci,} \Ch{Emi}{pištěli dudáci,}\NL
   To by \Ch{Dmi}{se krásně} \Ch{G}{žilo,} to \Ch{C}{by byl ráj.}
\SLOKA Kde není muzika, tam duše naříká,\NL
   tam je neveselo, tam je truchlivo.\NL
   Chtěla bych dvůr pestrý, kde znějí orchestry,\NL
   pěkně na veselo, žádné truchlivo.
\REFRENHRAJ
\end{TEXT}
