\begin{TEXT}{Kdyby se v komnatách}
\SLOKA \Ch{C}{K životu} na zámku,  \Ch{g}{mám} jednu poznámku\NL
   \Ch{F}{Je tu neveselo,} \Ch{C}{je tu truchlivo}\NL
   V ostatních královstvích \Ch{g}{ne}ní tak málo smích\NL
   \Ch{F}{Není} neveselo, není \Ch{C}{tru}chlivo
\REFREN \Ch{C7}{Kdyby} se \Ch{F}{v ko}mnatách, běhoun jak \Ch{e}{hrom} natáh\NL
   A na něm \Ch{d}{ak}robati, \Ch{G}{za}čali \Ch{C}{kej}klovati\NL
   \Ch{C7}{Kdyby} \Ch{F}{nám v paláci,} pištěli \Ch{e}{du}dáci\NL
   To by se \Ch{d}{krá}sně žilo, \Ch{G}{to} by byl \Ch{C}{ráj}
\SLOKA Kde není muzika, tam duše naříká,\NL
   tam je neveselo, tam je truchlivo.\NL
   Chtěla bych dvůr pestrý, kde znějí orchestry,\NL
   pěkně na veselo, žádné truchlivo.
\REFRENHRAJ
\end{TEXT}
