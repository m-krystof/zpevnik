\begin{TEXT}{Babylón a my (hymna 2015)}
\SLOKA[]\Ch{G}{Někdy} se věci na světě bůhví proč divně \Ch{C}{dějí}\NL
Octneš se v \Ch{G}{zajetí} jedna dvě tři a začneš se \Ch{D}{ptát}\NL
Bude to \Ch{G}{navždycky} co dělat mám a co po mně \Ch{C}{chtějí}\NL
Viny tě \Ch{G}{dostanou} a tak nezbývá než se \Ch{D}{kát}\NL
I když ten \Ch{G}{Babylón} kolem nás zdá se nám pořád \Ch{C}{cizí}\NL
A výhled z \Ch{G}{okna} není zrovna zlatý Jeruza\Ch{D}{lém}\NL
Každý den \Ch{G}{platí} i tady že \Ch{G7}{cenu má} to co je \Ch{C}{ryzí}\,\,\,\Ch{c}{ }\NL
A navzdory \Ch{G}{okolí} zvykům a \Ch{D}{tmám} hledat to \Ch{G}{své}\NL
\SLOKA[]\Ch{G}{Dál} \Ch{D}{dál} \Ch{C}{i v cizí} \Ch{D}{zemi}\NL
\Ch{G}{Dál} \Ch{D}{dál} žít se \Ch{C}{má}\NL
\Ch{G}{Dál} \Ch{D}{dál} \Ch{C}{nad} bohy \Ch{D}{všemi}\NL
\Ch{C}{Hospodin} \Ch{D}{náš Bůh} je \Ch{G}{Pán}
\end{TEXT}
