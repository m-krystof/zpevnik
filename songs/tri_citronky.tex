\begin{TEXT}{Tři citrónky}
\SLOKA \Ch{C}{V jedné} \Ch{a}{mořské} \Ch{d}{pusti}\Ch{G}{ně} 
\Ch{C}{ztroskotal} \Ch{a}{parník} \Ch{d}{v hlubi}\Ch{G}{ně,}
\Ch{C}{jenom tři} \Ch{a}{malé} \Ch{d}{citrón}\Ch{G}{ky} 
zůstaly na hladi\Ch{C}{ně}

\REFREN /: rýbaroba, rýbaroba, rýbaroba ču ču, :/(3x)
zůstaly na hladině

\SLOKA Jeden z nich povídá: přátelé,
netvařte se tak kysele
vždyt' je to přece veselé, 
že nám patří moře celé

\REFREN Ref.

\SLOKA A tak se citrónky plavily dál,
jeden jim k tomu na kytaru hrál,
a tak se plavily do dáli,
až na ostrov korálový

\REFREN Ref.

\SLOKA Tam je však stihla nehoda zlá,
byla to mořská příšera,
sežrala citrónky i s kůrou,
zakončila tak baladu mou

\REFREN Ref.

\end{TEXT}